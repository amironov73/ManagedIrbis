\chapter{Описание классов}

\section{Класс ManagedIrbis}

ManagedIrbis – «рабочая лошадка». Этот класс осуществляет связь с сервером, всю необходимую «перепаковку» данных и прочее и прочее. Собственно, это и есть управляемый клиент ИРБИС64.

Экземпляр клиента создаётся конструктором по умолчанию:
\begin{lstlisting}
var client = new ManagedIrbis ();
\end{lstlisting}

\subsection{Подключение к серверу}

Параметры подключения к серверу определяются следующими свойствами:

\begin{lstlisting}
public string Host { get; set; }
public int Port { get; set; }
public string Username { get; set; }
public string Password { get; set; }
public string Database { get; set; }
public IrbisWorkstation Workstation { get; set; }
\end{lstlisting}

\subsection{Работа с базой данных}

Имя текущей базы данных (каталога), с которой работает клиент, хранится в поле Database: 
public string Database { get; set; }
Кроме того, имеются два полезных метода: 
/// Временно устанавливает новое имя текущей базы данных.
/// Запоминает, к какой базе был подключен
/// клиент на момент смены.
/// Возвращает имя предыдущей текущей базы данных.
public string PushDatabase(string newDatabase);

/// Восстанавливает подключение к предыдущей базе данных,
/// сменённой методом PushDatabase().
/// Возвращает имя базы данных, к которой был подключен 
/// клиент на момент восстановления состояния.
public string PopDatabase()
Оба метода работают по принципу стека: предыдущие базы данных запоминаются в стеке и постепенно возвращаются по мере вызова метода PopDatabase(). Пример: 
// Мы работали с базой IBIS, 
// но решили временно подключиться к RDR
client.PushDatabase ("RDR"); // IBIS запоминается в стеке
IrbisRecord reader = client.SearchReadOneRecord ("I=1234");

// Теперь временно подключаемся к RQST
client.PushDatabase ("RQST"); // RDR также запоминается в стеке
IrbisRecord request = new IrbisRecord ();
...
client.WriteRecord (request, false, true); // Запись пойдёт в RQST

...
// Возвращаемся к базе RDR
client.PopDatabase ();
client.WriteRecord (reader, false, true); // Запись пойдёт в RDR

...
// Возвращаемся к исходной базе IBIS
client.PopDatabase ();
Рекомендую временные переключения между базами оформлять в блоке try-finally: 
client.PushDatabase ("CMPL");
try
{
   // Какие-то манипуляции с базой
}
finally
{
  // Гарантированно возвращаемся в правильный контекст
  // работы с базой данных
  client.PopDatabase();
}