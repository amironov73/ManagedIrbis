% !TEX TS-program = xelatex
% !TEX encoding = UTF-8 Unicode

\documentclass[a4paper,twoside,12pt]{book}
%\usepackage[utf8]{inputenc}
\usepackage{fontspec}
\usepackage{graphicx}
\usepackage[russian]{babel}
\usepackage{hyperref}
%\hypersetup{pdftex,colorlinks=true,allcolors=blue}
\usepackage{hypcap}
\usepackage{color}
\usepackage{listings}
\usepackage{multirow}

\graphicspath{ {Pics/} }

% Times New Roman
\setromanfont[
BoldFont=timesbd.ttf,
ItalicFont=timesi.ttf,
BoldItalicFont=timesbi.ttf,
]{times.ttf}
% Arial
\setsansfont[
BoldFont=arialbd.ttf,
ItalicFont=ariali.ttf,
BoldItalicFont=arialbi.ttf
]{arial.ttf}
% Courier New
\setmonofont[Scale=0.90,
BoldFont=courbd.ttf,
ItalicFont=couri.ttf,
BoldItalicFont=courbi.ttf,
Color={0019D4}
]{cour.ttf}

\begin{document}

\author{Алексей Миронов}
\title{Программирование\\
для АБИС "ИРБИС64"\\
в среде .NET Framework}
\date{Июль 2016}

\frontmatter
\maketitle

\lstset{ %
	language=[Sharp]C,                 % выбор языка для подсветки (здесь это С#)
	basicstyle=\small\ttfamily, % размер и начертание шрифта для подсветки кода
	numbers=left,               % где поставить нумерацию строк (слева\справа)
	numberstyle=\small,          % размер шрифта для номеров строк
	stepnumber=1,                % размер шага между двумя номерами строк
	numbersep=7pt,      % как далеко отстоят номера строк от подсвечиваемого кода
	backgroundcolor=\color{white}, % цвет фона подсветки 
	% используем \usepackage{color}
	showspaces=false,        % показывать или нет пробелы специальными отступами
	showstringspaces=false,    % показывать или нет пробелы в строках
	showtabs=false,            % показывать или нет табуляцию в строках
	frame=single,              % рисовать рамку вокруг кода
	tabsize=2,                % размер табуляции по умолчанию равен 2 пробелам
	captionpos=t,              % позиция заголовка вверху [t] или внизу [b] 
	breaklines=true,           % автоматически переносить строки (да\нет)
	breakatwhitespace=false,   % переносить строки только если есть пробел
	escapeinside={\%*}{*)},     % если нужно добавить комментарии в коде
	keywordstyle=\color{blue}, % стиль (цвет) зарезервированных слов
	stringstyle=\color{red}   % стиль (цвет) строковых литералов
}

\clearpage
\thispagestyle{empty}
	Описаны возможности расширения АБИС "ИРБИС64",
	серверный протокол и фреймворк ManagedIrbis,
	позволяющий создавать приложения произвольной
	сложности на основе "ИРБИС64".
	
	Для автоматизаторов библиотек и программистов,
	создающих решения для "ИРБИС64".

\tableofcontents

\mainmatter
\chapter*{Введение}
\addcontentsline{toc}{chapter}{Введение}
\chaptermark{Введение}

Фреймворк «ManagedIrbis» предназначен для организации программного доступа к ресурсам, находящихся под управлением сервера ИРБИС64, и может использоваться для как для расширения функциональности стандартных АРМ, входящих в поставку АБИС ИРБИС64, так и для создания собственных программных продуктов, совместимых с ИРБИС64.

\section*{Совместимость}

Фреймворк совместим со следующими версиями ИРБИС64:
2004	2005	2006	2007	2008	2009	2010	2011	2012	2013	2014	2015

Совместимость с конкретной версией сервера ИРБИС64 устанавливается по результатам прогона набора стандартных тестов: подключение к серверу, получение служебной информации (версия сервера, количество лицензий и т. д.), чтение записей, форматирование записей, сохранение записей и т. д.

\section*{Инструментарий}

Фреймворк написан на языке C\# в среде Microsoft Visual Studio 2013 для Micorsoft .NET Framework 4.5. Для сборки библиотеки из исходных текстов необходим совместимый инструментарий: Visual Studio 2013 или более новой версии как бесплатной редакции (Express), так и платной (Standard, Profes\-sional и т. д.).

Библиотека должна без модификации успешно собираться средами Mono\-Develop (версия не ниже 4.0) и Sharp\-Develop (версия не ниже 4.4).
Однако всё многообразие альтернативного инструментария не было протестировано авторами (и они не ставили перед собой подобной задачи), и авторы рекомендуют использовать для сборки Visual Studio 2013.

\section*{Системные требования}

Основным системным требованием библиотеки является наличие Microsoft.Net framework 4.5/4.5.1/4.5.2 или совместимой с ним среды исполнения управляемого кода.
Фреймворк должен функционировать в следующем окружении:

\begin{table}[htbp]
	\centering
	\caption{Поддерживаемые окружения}
	\begin{tabular}{ | p{0.4\textwidth} | p{0.4\textwidth} | }
	\hline
	\textbf{Окружение} & 
	\textbf{Функционирование, требования}
	\\
	\hline
	\hline
	Microsoft Windows XP & Не поддерживается \\
	\hline
	Microsoft Windows Vista SP2 & Необходимо установить .Net framework 4.5 \\
	\hline
	Microsoft Windows Server 2003 & Не поддерживается \\
	\hline
	Microsoft Windows 7 SP1 & Необходимо установить .Net framework 4.5 \\
	\hline
	Microsoft Windows Server 2008 SP2/2008 R2 SP1 & Необходимо установить .Net framework 4.5 \\
	\hline
	Microsoft Windows Server 2012/2012 R2 & Предустановлен в операционной системе \\
	\hline
	\end{tabular}
\end{table}

\section*{ManagedIrbis в Интернет}

Исходные коды фреймворка размещены на Git-хостинге github.com по адресу https://github.com/amironov73/arsmagna. Доступ к репозиторию открыт на чтение для всех.
Исполняемые файлы фреймворка опубликованы на сервисе NuGet по адресу

\section*{Лицензия}

Фреймворк распространяется как продукт с открытым исходным кодом. Любой желающий может:

\begin{itemize}
	\item Использовать бинарный релиз библиотеки в своих проектах в неизменном виде – в этом случае требуется лишь указание на авторство библиотеки.
	\item Адаптировать исходный код для собственных нужд и использовать в своих проектах модифицированную версию библиотеки или [модифицированные] фрагменты кода из неё – в этом случае требуется указание на авторство библиотеки и факт модификации её кода.
\end{itemize}

Никаких лицензионных отчислений в вышеперечисленных случаях не требуется. 

\section*{Благодарности}

Авторы выражают благодарность:

\begin{itemize}
	\item \textbf{Ивану Батраку} (СФУ), протестировавшему библиотеку на совместимость со старыми версиями ИРБИС-сервера;
	\item \textbf{Арсению Валентиновичу Шувалову} (Саратовская государственная консерватория им. Л. В. Собинова), выявившему ошибки в библиотеке;
	\item \textbf{Артёму Васильевичу Гончарову} (Научная музыкальная библиотека Санкт-Петер\-бургской Консерватории им. Н. А. Римского-Корсакова), выявившему некоторые досадные ошибки в библиотеке.
\end{itemize}

\section*{Версии и совместимость}

Данное руководство описывает версию 1.3.0.24 библиотеки. Версия библиотеки физически хранится как ресурс VERSION сборки ManagedClient.dll и как статическое свойство Version класса ManagedClient64.

Подробнее о проверке версий см. пункт «Определение версии сервера и клиента».
На данный момент несовместимых версий библиотеки нет, поэтому обновление может осуществляться простым копированием новой сборки поверх старой.
Все будущие несовместимости, если таковые появятся, будут описаны в данном разделе.



\chapter{ИРБИС64-сервер и его окружение}

\begin{description}
	\item[Сервер] это сервер
	\item[Клиент] это клиент
\end{description}

У попа была собака. Он её любил. Она съела кусок мяса. Он её убил. В землю закопал. Надпись написал. У попа была собака. Он её любил. Она съела кусок мяса. Он её убил. В землю закопал. Надпись написал.

У попа была собака. Он её любил. Она съела кусок мяса. Он её убил. В землю закопал. Надпись написал. У попа была собака. Он её любил. Она съела кусок мяса. Он её убил. В землю закопал. Надпись написал. У попа была собака. Он её любил. Она съела кусок мяса. Он её убил. В землю закопал. Надпись написал.

У попа была собака. Он её любил. Она съела кусок мяса. Он её убил. В землю закопал. Надпись написал.

У попа была собака. Он её любил. Она съела кусок мяса. Он её убил. В землю закопал. Надпись написал.

У попа была собака. Он её любил. Она съела кусок мяса. Он её убил. В землю закопал. Надпись написал. У попа была собака. Он её любил. Она съела кусок мяса. Он её убил. В землю закопал. Надпись написал.
\chapter{Протокол ИРБИС64}

У попа была собака. Он её любил. Она съела кусок мяса. Он её убил. В землю закопал. Надпись написал. У попа была собака. Он её любил. Она съела кусок мяса. Он её убил. В землю закопал. Надпись написал.

У попа была собака. Он её любил. Она съела кусок мяса. Он её убил. В землю закопал. Надпись написал. У попа была собака. Он её любил. Она съела кусок мяса. Он её убил. В землю закопал. Надпись написал. У попа была собака. Он её любил. Она съела кусок мяса. Он её убил. В землю закопал. Надпись написал.

У попа была собака. Он её любил. Она съела кусок мяса. Он её убил. В землю закопал. Надпись написал.

У попа была собака. Он её любил. Она съела кусок мяса. Он её убил. В землю закопал. Надпись написал.

У попа была собака. Он её любил. Она съела кусок мяса. Он её убил. В землю закопал. Надпись написал. У попа была собака. Он её любил. Она съела кусок мяса. Он её убил. В землю закопал. Надпись написал.
\chapter{ManagedIrbis. Быстрый старт}

У попа была собака. Он её любил. Она съела кусок мяса. Он её убил. В землю закопал. Надпись написал. У попа была собака. Он её любил. Она съела кусок мяса. Он её убил. В землю закопал. Надпись написал.

У попа была собака. Он её любил. Она съела кусок мяса. Он её убил. В землю закопал. Надпись написал. У попа была собака. Он её любил. Она съела кусок мяса. Он её убил. В землю закопал. Надпись написал. У попа была собака. Он её любил. Она съела кусок мяса. Он её убил. В землю закопал. Надпись написал.

У попа была собака. Он её любил. Она съела кусок мяса. Он её убил. В землю закопал. Надпись написал.

У попа была собака. Он её любил. Она съела кусок мяса. Он её убил. В землю закопал. Надпись написал.

У попа была собака. Он её любил. Она съела кусок мяса. Он её убил. В землю закопал. Надпись написал. У попа была собака. Он её любил. Она съела кусок мяса. Он её убил. В землю закопал. Надпись написал.
\chapter{Описание классов}

\section{Класс ManagedIrbis}

ManagedIrbis – «рабочая лошадка». Этот класс осуществляет связь с сервером, всю необходимую «перепаковку» данных и прочее и прочее. Собственно, это и есть управляемый клиент ИРБИС64.

Экземпляр клиента создаётся конструктором по умолчанию:
\begin{lstlisting}
var client = new ManagedIrbis ();
\end{lstlisting}

\subsection{Подключение к серверу}

Параметры подключения к серверу определяются следующими свойствами:

\begin{lstlisting}
public string Host { get; set; }
public int Port { get; set; }
public string Username { get; set; }
public string Password { get; set; }
public string Database { get; set; }
public IrbisWorkstation Workstation { get; set; }
\end{lstlisting}

\subsection{Работа с базой данных}

Имя текущей базы данных (каталога), с которой работает клиент, хранится в поле Database:
\begin{lstlisting}
public string Database { get; set; }
\end{lstlisting}
Кроме того, имеются два полезных метода:
\begin{lstlisting}
/// Временно устанавливает новое имя текущей базы данных.
/// Запоминает, к какой базе был подключен
/// клиент на момент смены.
/// Возвращает имя предыдущей текущей базы данных.
public string PushDatabase(string newDatabase);

/// Восстанавливает подключение к предыдущей базе данных,
/// сменённой методом PushDatabase().
/// Возвращает имя базы данных, к которой был подключен 
/// клиент на момент восстановления состояния.
public string PopDatabase()
\end{lstlisting}
Оба метода работают по принципу стека: предыдущие базы данных запоминаются в стеке и постепенно возвращаются по мере вызова метода PopDatabase(). Пример:
\begin{lstlisting}
// Мы работали с базой IBIS, 
// но решили временно подключиться к RDR
client.PushDatabase ("RDR"); // IBIS запоминается в стеке
IrbisRecord reader = client.SearchReadOneRecord ("I=1234");

// Теперь временно подключаемся к RQST
client.PushDatabase ("RQST"); // RDR также запоминается в стеке
IrbisRecord request = new IrbisRecord ();
...
client.WriteRecord (request, false, true); // Запись пойдёт в RQST

...
// Возвращаемся к базе RDR
client.PopDatabase ();
client.WriteRecord (reader, false, true); // Запись пойдёт в RDR

...
// Возвращаемся к исходной базе IBIS
client.PopDatabase ();
Рекомендую временные переключения между базами оформлять в блоке try-finally: 
client.PushDatabase ("CMPL");
try
{
   // Какие-то манипуляции с базой
}
finally
{
  // Гарантированно возвращаемся в правильный контекст
  // работы с базой данных
  client.PopDatabase();
}
\end{lstlisting}

\subsection{Последовательный поиск}

В простейшем случае этот поиск состоит из указания двух выражений: 1) обычное поисковое выражение, отбирающее записи по словарю, 2) булево выражение, которое будет применено к каждой найденной записи. 

Пример такого поиска: сначала мы отбираем все книги с фамилией автора "Пушкин", а затем проверяем, не содержит ли поле 200 буквосочетания "сказк": 
\begin{lstlisting}
// Выполняем последовательный поиск
int[] found = Client.SequentialSearch
    (
        "\"A=Пушкин$\"", // отбор по словарю
        "v200:'сказк'"   // булево выражение
    );

// Выводим найденные записи на консоль
foreach (int mfn in found)
{
    string description = Client.FormatRecord
        (
            "@brief",
            mfn
        );
    Console.WriteLine(description);
}
\end{lstlisting}
На экран будет выведено что-то вроде: 
\begin{verbatim}
Пушкин, Александр Сергеевич. Сказка о царе Салтане, о сыне его славном и могучем богатыре князе Гвидоне Салтановиче и о прекрасной царевне Лебеди / А.С. Пушкин ; Послесл. М. Сокольникова; Ил. А.М. Куркина (Палех), 1972. - 39 c. 

Пушкин, Александр Сергеевич. Сказки / А.С. Пушкин; Ил. А. Кокорина, 1976. - 72 с. 

Пушкин, Александр Сергеевич. Собрание сочинений : В 10 т. Т.3. : Поэмы. Сказки, 1982. - 671 с. 

Пушкин, Александр Сергеевич. Сказка о царе Салтане,о сыне его славном и могучем богатыре князе Гвидоне Салтановиче и о прекрасной царевне Лебеди / А.С. Пушкин, 1971. - 94 с. 

Пушкин, Александр Сергеевич. Сочинения : в 2 т. Т. 1 : Стихотворения, поэмы, сказки, 1982. - 365 с. 
\end{verbatim}

Есть вариант этого же метода, предоставляющий больший контроль: 
\begin{lstlisting}
Client.SequentialSearch
(
"\"A=Пушкин$\"",
100,   // запрашиваемое количество записей
1,     // номер первой записи (смещение)
10000, // минимальный MFN
20000, // максимальный MFN
"v200:'сказк'"        
);
\end{lstlisting}
Если вместо "\"A=Пушкин\$\"" передать null, то будет выполнен последовательный поиск по всей базе данных, что, скорее всего, создаст большую нагрузку на сервер. 

Подробнее см. документацию на протокол ИРБИС": http://sntnarciss.ru/irbis/spravka/wtcp007007020.htm.
\chapter*{Библиотека AM.Core}
\addcontentsline{toc}{chapter}{Библиотека AM.Core}

\backmatter
% bibliography, glossary and index would go here.

\listoffigures
\addcontentsline{toc}{chapter}{Список иллюстраций}
\listoftables
\addcontentsline{toc}{chapter}{Список таблиц}

\end{document}